\documentclass[a4paper, 11pt]{article}

\usepackage[british]{babel}
\usepackage[autostyle]{csquotes}
\usepackage[colorlinks=true, urlcolor=blue, citecolor=blue]{hyperref}
\usepackage{graphicx}
\usepackage{float}

\begin{document}

\title{Borůvka's Algorithm}
\author{Student Number: 690065435}
\date{December 2022}
\maketitle

\section{Principles of Borůvka's Algorithm}
Borůvka's algorithm is a greedy algorithm that finds a minimum spanning tree for a connected, edge-weighted undirected graph. It originated from Otakar Borůvka in 1926, as a method of constructing an efficient electricity network for Moravia, a region of the Czech Republic \cite{nevsetvril2001otakar}. It was independently rediscovered by numerous other researchers in later years, most notably by Georges Sollin in 1965, which has led to the algorithm also being known as Sollin's algorithm in parallel computing literature \cite{sollin1965trace}.

The algorithm uses a divide-and-conquer approach that is based on the idea of building a forest of trees. At each step, it finds the cheapest edge that connects two different trees and combines the trees into a single tree. The algorithm continues until there is only one tree left, which is the minimum spanning tree.

\section{Pseudocode}

\section{Time and Space Complexity Analysis}

\section{Limitations and Constraints}

\section{Real-World Applications}

\bibliography{main}
\bibliographystyle{ieeetr.bst}

\end{document}