\documentclass[a4paper, 11pt]{article}

\usepackage[british]{babel}
\usepackage[autostyle]{csquotes}
\usepackage[colorlinks=true, urlcolor=blue, citecolor=blue]{hyperref}
\usepackage{graphicx}
\usepackage{float}

\begin{document}

\title{Boruvka's Algorithm}
\author{Student Number: 690065435}
\date{December 2022}
\maketitle

\section{Principles of Boruvka's Algorithm}
Boruvka's algorithm (also known as Sollin's algorithm) is a greedy algorithm that finds a minimum spanning tree for a connected, edge-weighted undirected graph. It is a divide-and-conquer algorithm that is based on the idea of building a forest of trees. At each step, it finds the cheapest edge that connects two different trees and combines the trees into a single tree. The algorithm continues until there is only one tree left, which is the minimum spanning tree.

\section{Pseudocode}

\section{Time and Space Complexity Analysis}

\section{Limitations and Constraints}

\section{Real-World Applications}

\bibliography{main}
\bibliographystyle{ieeetr.bst}

\end{document}